\documentclass[12pt,a4paper]{article}

% Пакеты для русского языка
\usepackage[utf8]{inputenc}
\usepackage[T2A]{fontenc}
\usepackage[russian]{babel}

% Пакеты для форматирования
\usepackage[margin=2cm]{geometry}
\usepackage{amsmath, amsfonts, amssymb}
\usepackage{graphicx}
\usepackage{enumitem}
\usepackage{titlesec}
\usepackage{fancyhdr}

% Настройка заголовков
\titleformat{\section}{\large\bfseries}{\thesection.}{1em}{}
\titleformat{\subsection}{\normalsize\bfseries}{\thesubsection.}{1em}{}

% Настройка колонтитулов
\pagestyle{fancy}
\fancyhf{}
\fancyhead[L]{ {{work_name}} }
\fancyhead[R]{\thepage}
\fancyfoot[C]{Всего вариантов: {{ total_variants }}}

% Настройка отступов
\setlength{\parindent}{0pt}
\setlength{\parskip}{6pt}

\begin{document}

% Титульный лист
\begin{center}
    {\LARGE\bfseries {{ work_name }} }\\[0.8cm]
    
    {\large Время выполнения: {{ work.duration }} минут}\\[0.4cm]
    
    {\large Количество вариантов: {{ total_variants }}}\\[0.3cm]
    {\normalsize \today}
\end{center}

\vspace{1cm}

% Общая инструкция
{\small\textit{%
Внимательно прочитайте задания и дайте развернутые ответы. \\
Показывайте ход решения. Каждый вариант начинается с новой страницы.
}}

\newpage

% Перебираем все варианты


% Заголовок варианта
\begin{center}
    {\Large\bfseries Вариант {{ variant_data.variant.number }} }
\end{center}

\vspace{0.8cm}

% Задания варианта

\section*{Задание {{ task_data.number }}}

% НОВОЕ: Используем предварительно сгенерированный LaTeX код с minipage
{{ task_data.latex_content|safe }}

% ИСПРАВЛЕНИЕ 3: Заменяем большой пробел на малый
\vspace{1ex}



% Новая страница для следующего варианта (кроме последнего)

\newpage





\newpage
% Лист ответов
\section*{\centering ОТВЕТЫ}


\subsection*{Вариант {{ variant_data.variant.number }}}

\begin{enumerate}

\item {{ task_data.answer }}

\end{enumerate}

\vspace{0.5cm}




\end{document}
